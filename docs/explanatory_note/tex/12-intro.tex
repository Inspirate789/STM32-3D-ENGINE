\Introduction


В современном мире компьютерная графика является неотъемлемой частью нашей жизни. Сегодня она используется в различных сферах 
жизнедеятельности человека, таких как визуализация данных, компьютерные игры, кинематограф и многие другие. Поэтому перед людьми, 
создающими трехмерные сцены, встает задача создания реалистичных изображений, которые будут учитывать оптические явления преломления, 
отражения и рассеивания света, свойства поверхностей, тени, а также восприятие окружающего мира человеческим глазом.

Когда встаёт задача изображения трёхмерной сцены на конкретной вычислительной машине, для каждой платформы нередко приходится разрабатывать 
отдельные программные решения, учитывающие особенности аппаратного обеспечения. Одной из платформ, которая нуждается в программном 
обеспечении, решающем задачи компьютерной графики, является архитектура ARM и в частности семейство микроконтроллеров STM32. Такая 
необходимость возникает вследствие роста популярности микроконтроллеров этого семейства по всему миру, в том числе и в России, а также 
расширения области их промышленного применения. Уже сейчас STM32 используется в отраслях, в которых необходимо решать задачи компьютерной 
графики, начиная с промышленной автоматики и заканчивая пользовательской электроникой и устройствами интернета вещей.

Цель данной работы – реализовать программный инструмент для построения моделей трехмерных объектов, ориентированный на микроконтроллеры 
семейства STM32.

Чтобы достигнуть поставленной цели, требуется решить следующие задачи:
\begin{enumerate}
	\item[1)] описать структуру трехмерной сцены, включая объекты, из которых состоит сцена, и определить способ задания исходных данных;
	\item[2)] выбрать и адаптировать существующие алгоритмы трехмерной графики, позволяющие визуализировать трехмерную сцену, для выполнения на заданном оборудовании;
	\item[3)] реализовать выбранные алгоритмы построения трехмерной сцены;
	\item[4)] исследовать возможности микроконтроллеров семейства STM32 для их применения при решении задач компьютерной графики. 
\end{enumerate}

%\begin{itemize}
%\item проанализировать существующую всячину;
%\item спроектировать свою, новую всячину;
%\item изготовить всякую всячину;
%\item проверить её работоспособность.
%\end{itemize}
